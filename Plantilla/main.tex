%%% PLANTILLA DISEÑADA PARA LA REALIZACIÓN DE TESIS DE GRADO DE LA UNIVERSIDAD NACIONAL DEL CALLAO
%% AUTOR DE LA PLANTILLA: RODOLFO ZEVALLOS SALAZAR

\documentclass[12pt,a4paper,oneside]{report}

\usepackage[T1]{fontenc}
\usepackage{times}
\usepackage[utf8]{inputenc}
\usepackage{amsmath}
\usepackage{graphicx}
\usepackage{multicol}
\usepackage{longtable}
\usepackage[refpages]{gloss}
\usepackage{float}
\usepackage{anysize}
\usepackage{appendix}
\usepackage{lscape} 
\usepackage{pdflscape}
\usepackage{multirow}
\usepackage{listings}
\usepackage{color}
\usepackage{setspace}
\usepackage{enumerate} 


\begin{document}

%----------------------------------------------------------------------------------------
%	CONFIGURACION
%----------------------------------------------------------------------------------------

\marginsize{3.0cm}{3.0cm}{4.0cm}{3.0cm}
\renewcommand*{\contentsname}{ÍNDICE}
\renewcommand*{\listtablename}{Índice de tablas}
\renewcommand*{\listfigurename}{Índice de figuras}
\renewcommand{\baselinestretch}{1.0}
\renewcommand{\appendixname}{Anexos}
\renewcommand{\appendixtocname}{Anexos}
\renewcommand{\appendixpagename}{Anexos}
\renewcommand{\thetable}{\arabic{chapter}.\arabic{table}}
\renewcommand*{\tablename}{Tabla}
\renewcommand*{\chaptername}{Capítulo}
\renewcommand*{\thechapter}{\Roman{chapter}}
\renewcommand{\thesection}{\arabic{chapter}.\arabic{section}}
\renewcommand{\figurename}{Figura}
\renewcommand{\thefigure}{\arabic{chapter}.\arabic{figure}}
\renewcommand{\theequation}{\arabic{chapter}.\arabic{equation}}

%----------------------------------------------------------------------------------------
%	PORTADA
%----------------------------------------------------------------------------------------

\begin{titlepage}
 
\begin{center}
 
 {\huge \bf UNIVERSIDAD NACIONAL DEL CALLAO}\\
 
{\Large FACULTAD DE INGENIERÍA INDUSTRIAL Y DE SISTEMAS}\\{\Large ESCUELA PROFESIONAL DE INGENIERÍA DE SISTEMAS}\\[2.0cm]


\begin{center}
\includegraphics[width=0.4\textwidth]{imagenes/logo_unac}
\end{center}

\vspace{1cm}
\title{} % titulo de tu tesis para latex
{\bf \large . }\\[1cm] % titulo de tu tesis

{TESIS PARA OPTAR EL TÍTULO PROFESIONAL DE INGENIERO DE SISTEMAS}\\[1.5cm]
 
{{\bf Autor}: }\\[2.0cm] % nombres del autor o autores

{\large Callao, Julio, 201-}\\[0.2cm] % fecha de sustentación
{PERÚ}
\end{center}

\end{titlepage}

\newpage
\newpage
%----------------------------------------------------------------------------------------
%	Dedicatoria
%----------------------------------------------------------------------------------------

\begin{titlepage}

\begin{flushright}
{\large \bf DEDICATORIA}

\cite{arceda2017deteccion}asd
\cite{En2017}
\\
\textit{}
\\
\textit{} % agregar tu dedicatoria
\end{flushright}
\end{titlepage}

%----------------------------------------------------------------------------------------
%	Agradecimientos
%----------------------------------------------------------------------------------------

\begin{titlepage}

\begin{flushright}
{\large \bf AGRADECIMIENTO}
\\
\textit{}
{} % agregar tu dedicatoria
\end{flushright}
\end{titlepage}

%----------------------------------------------------------------------------------------
%	TABLA DE CONTENIDOS
%---------------------------------------------------------------------------------------

\tableofcontents
\cleardoublepage
\listoftables
\listoffigures 
\makegloss
\newpage

%----------------------------------------------------------------------------------------
%	Resumen
%----------------------------------------------------------------------------------------

\chapter*{\centering \large RESUMEN} % si no queremos que añada la palabra "Capitulo"
\addcontentsline{toc}{section}{RESUMEN} % si queremos que aparezca en el índice
\markboth{RESUMEN}{RESUMEN} % encabezado
\doublespacing

%----------------------------------------------------------------------------------------
%	Abstract
%----------------------------------------------------------------------------------------

\chapter*{\centering \large ABSTRACT}
\addcontentsline{toc}{section}{ABSTRACT}
\markboth{ABSTRACT}{ABSTRACT}


%----------------------------------------------------------------------------------------
%	INTRODUCCIÓN
%----------------------------------------------------------------------------------------

\chapter*{\centering \large INTRODUCCIÓN} % si no queremos que añada la palabra "Capitulo"
\addcontentsline{toc}{section}{INTRODUCCIÓN} % si queremos que aparezca en el índice
\markboth{INTRODUCCIÓN}{INTRODUCCIÓN} % encabezado

%----------------------------------------------------------------------------------------
%	PLANTEAMIENTO DE LA INVESTIGACIÓN
%----------------------------------------------------------------------------------------

\chapter{PLANTEAMIENTO DE LA INVESTIGACIÓN}

\section{Identificación del problema}

\section{Formulación del problema}

\subsection{General}

\subsection{Específicos}

\section{Objetivos de la investigación}

\subsection{General}

\subsection{Específicos}

\section{Justificación}

\section{Importancia}


%----------------------------------------------------------------------------------------
%	MARCO TEORICO
%----------------------------------------------------------------------------------------

\chapter{MARCO TEÓRICO}

\section{Antecedentes de estudio}

\section{Marco conceptual}


%----------------------------------------------------------------------------------------
%	VARIABLES E HIPÓTESIS
%----------------------------------------------------------------------------------------

\chapter{VARIABLES E HIPÓTESIS}

\section{Variables de la investigación}

\subsection{Variable dependiente}

\subsection{Variable independiente}

\section{Operacionalización de variables}

\section{Hipótesis de la investigación}

\subsection{General}

\subsection{Específicos}

%----------------------------------------------------------------------------------------
%	METODOLOGÍA
%----------------------------------------------------------------------------------------
\chapter{METODOLOGÍA}

\section{Tipo de investigación}

\section{Diseño de la investigación}

\section{Población y muestra}

\section{Técnicas e instrumentos de recolección de datos}

\subsection{Técnicas de muestreo}

\subsection{Recopilación de datos}

\subsection{Procesamiento}

\section{Procedimientos de recolección de datos}

\section{Procesamiento estadístico y análisis de datos}

%----------------------------------------------------------------------------------------
%	RESULTADOS
%----------------------------------------------------------------------------------------

\chapter{RESULTADOS}

%----------------------------------------------------------------------------------------
%	DISCUCIÓN DE RESULTADOS
%----------------------------------------------------------------------------------------

\chapter{DISCUSIÓN DE RESULTADOS}

\section{Constratación de hipótesis con los resultados}

\section{Contrastación de resultados con otros estudios similares}

%----------------------------------------------------------------------------------------
%	CONCLUSIONES
%----------------------------------------------------------------------------------------

\chapter{ CONCLUSIONES}

\section{Conclusiones}


%----------------------------------------------------------------------------------------
%	RECOMENDACIONES
%----------------------------------------------------------------------------------------

\chapter{\hspace{0.23cm}RECOMENDACIONES}

\section{Recomendaciones}

%----------------------------------------------------------------------------------------
%	BIBLIOGRAFIA
%----------------------------------------------------------------------------------------

\bibliographystyle{acm} % estilo de la bibliografia
\renewcommand*{\bibname}{REFERENCIAS}
\bibliography{bibliografia} % nombre del archivo .bib
\addcontentsline{toc}{chapter}{REFERENCIAS}
\newpage


%----------------------------------------------------------------------------------------
%	ANEXOS
%----------------------------------------------------------------------------------------

\appendix
\clearpage
\addappheadtotoc
\appendixpage

\chapter{Matriz de consistencia}

\chapter{Anexo para respaldo de la investigación}

\end{document} 


